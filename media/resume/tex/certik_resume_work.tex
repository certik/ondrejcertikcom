\documentclass[11pt,letterpaper]{article}

\usepackage[utf8]{inputenc}
\pagestyle{empty}

\topmargin=0cm
\textwidth=16cm
\oddsidemargin=0cm
\evensidemargin=0cm
\textheight=23cm

%\renewcommand{\baselinestretch}{1}
%\setlength{\textheight}{8.875in}
%\setlength{\textwidth}{6.875in}
%\setlength{\columnsep}{0.3125in}
%\setlength{\topmargin}{0in}
%\setlength{\headheight}{0in}
%\setlength{\headsep}{0in}
%\setlength{\parindent}{1pc}
%\setlength{\oddsidemargin}{-.304in}
%\setlength{\evensidemargin}{-.304in}

\begin{document}

\begin{center}
\vbox{}
\vspace{-1.9cm}
{\Large \bf Biographical Sketch: Ondřej Čertík}\\
\vspace{4mm}
PhD student, Chemical Physics, University of Nevada, Reno, NV.
Phone: (510) 684-2858, e-mail: {\tt ondrej@certik.cz}.\\
Homepage: {\tt http://ondrejcertik.com/.}
\end{center}

\subsubsection*{Professional preparation}
\hspace{-3.4mm}
\begin{tabular}{lll}
Charles University (Prague, Czech Republic) & MMCM\thanks{Mathematical Modeling and Computational Mathematics} 
&  M.S. Degree, 1996\\
Charles University (Prague, Czech Republic) & MMCM & Ph.D. Degree, 1999 \\
Johannes-Kepler University (Linz, Austria) &  & Post-doc, 1999--2001 \\
The University of Texas at Austin (Austin, TX) &  & Post-doc, 2001--2002 \\
Rice University (Houston, TX) & & Post-doc, 2002--2004\\[3mm]
\end{tabular}

\noindent
${}^1$MMCM = Mathematical Modeling and Computational Mathematics. An interdisciplinary M.S. and Ph.D.
program combining Physics, Mathematics, and Computer Science.

\subsubsection*{Appointments}

Associate Professor, University of Nevada, Reno (UNR): January 2009 -- present \\
Associate Professor, University of Texas at El Paso (UTEP): Aug. 2004 -- Dec. 2008 

\subsubsection*{Publications related to the project}
\begin{enumerate}
\item P. Solin, O. Certik, S. Regmi: The FEMhub Project and Classroom Teaching of Numerical Methods. 
      In: Proc. of the 8th Python in Science Conference (SciPy 2009), Pasadena, 
      Aug. 2009 (http://hpfem.org/publications/papers/2009/scipy09\_paper-39.pdf).
\vspace{-3mm}
\item I. Dolezel, P. Karban, P. Solin: {\em Integral Methods in Low-Frequency 
      Electromagnetics}, 388 pages, J. Wiley \& Sons, 2009.
\vspace{-3mm}
\item P. Solin: {\em Partial Differential Equations and the 
      Finite Element Method}, 504 pages, J. Wiley \& Sons, 2005. 
\vspace{-3mm}
\item P. Solin, K. Segeth, I. Dolezel:
      {\em Higher-Order Finite Element Methods}, 408 pages, 
      Chapman \& Hall/ CRC Press, 2003.
\vspace{-3mm}
\item P. Solin, T. Vejchodsky: Higher-Order Finite Elements 
      Based on Generalized Eigenfunctions of the Laplacian: 
      Int. J. Numer. Methods Engrg. 73 (2007), 1374 - 1394.
%\vspace{-3mm}
%\item L. Dubcova, P. Solin, J. Cerveny, P. Kus: Space and Time Adaptive Two-Mesh $hp$-FEM 
%      for Transient Microwave Heating Problems. Electromagnetics, 2009, in press.
%\vspace{-3mm}
%\item P. Solin, D. Andrs, J. Cerveny, M. Simko: PDE-Independent Adaptive $hp$-FEM 
%      Based on Hierarchic Extension of Finite Element Spaces, J. Comput. Appl. Math., 2009, in press.
%\vspace{-3mm}
%\item P. Solin, J. Cerveny, I. Dolezel: Arbitrary-Level Hanging Nodes and 
%      Automatic Adaptivity in the $hp$-FEM, Math. Comput. Simul. 77 (2008), 117 - 132, 
%\vspace{-3mm}
\end{enumerate}

\vspace{-4mm}
\noindent
\subsubsection*{Other significant publications}
\begin{enumerate}
%\vspace{-2mm}
%\vspace{-8mm}
\item P. Solin, D. Andrs, J. Cerveny, M. Simko: PDE-Independent Adaptive $hp$-FEM 
      Based on Hierarchic Extension of Finite Element Spaces, J. Comput. Appl. Math., 2009, in press.
\vspace{-3mm}
\item L. Dubcova, P. Solin, J. Cerveny, P. Kus: Space and Time Adaptive Two-Mesh $hp$-FEM 
      for Transient Microwave Heating Problems. Electromagnetics, 2009, in press.
\vspace{-3mm}
\item P. Solin, J. Cerveny, L. Dubcova: Adaptive Multi-Mesh $hp$-FEM for Linear Thermoelasticity.
      J. Comput. Appl. Math, 2009, doi 10.1016/j.cam.2009.08.092.
\vspace{-3mm}
\item P. Solin, J. Cerveny, I. Dolezel: Arbitrary-Level Hanging Nodes and 
      Automatic Adaptivity in the $hp$-FEM, Math. Comput. Simul. 77 (2008), 117 - 132, 
\vspace{-3mm}
\item P. Solin, J. Cerveny, L. Dubcova, I. Dolezel: Multi-Mesh $hp$-FEM for Thermally 
      Conductive Incompressible Flow. In: Proceedings of ECCOMAS Coupled Problems 2007, 
      Ibiza Island, Spain, pp. 547 - 550.
%\vspace{-3mm}
%\item  P. Solin, L. Demkowicz: Goal-Oriented $hp$-Adaptivity 
%      for Elliptic Problems, Comput. 
%      Methods Appl. Mech. Engrg. 193, pp. 449 - 468, 2004. 
%\item T. Vejchodsky, P. Solin: Static Condensation, 
%      Orthogonalization of Bubble Functions, 
%      and ILU Preconditioning in the $hp$-FEM: J. Comput. Appl. Math. 218/1 (2008), 
%      192-200.
%\item P. Solin, J. Avila: Equidistributed Error Mesh for Problems 
%      with Exponential Boundary Layers, J. Comput. Appl. Math. 218/1 (2008), 157-166.
%\vspace{-3mm}
%\vspace{-2.5mm}
\end{enumerate}
%\newpage
%\vbox{}
%\vspace{-2.4cm}
\subsubsection*{Synergistic activities}

\begin{itemize}
\item {\em Innovations in teaching and training:} 
      The PI developed the {\em Online Numerical Methods Laboratory} ({\tt http://nb.femhub.org/}), 
      an innovative free online tool for undergraduate numerical methods courses. 
      %The online 
      %laboratory does not require any software or hardware, and is
      %freely available to instructors, students, and the general public via 
      %any web browser.
\item {\em Development and/or refinement of research tools:} 
      The PI has contributed significantly to the development
      of adaptive higher-order finite element methods ($hp$-FEM). In particular 
      he has developed novel higher-order shape functions, PDE-independent adaptivity 
      algorithms, $hp$-FEM with arbitrary-level hanging nodes, goal-oriented adaptive
      $hp$-FEM, multimesh $hp$-FEM for multiphysics coupled problems, and 
      space-time adaptive $hp$-FEM on dynamical meshes for time-dependent problems. 
%\vspace{-3mm}
\item {\em Integration and transfer of knowledge:} 
      The PI makes modern adaptive higher-order computational methods 
      available to practitioners and students through the open 
      source project Hermes ({\tt http://hpfem.org/hermes/}). 
      Another PI's open source project, FEMhub ({\tt http: //hpfem.org/femhub/}) 
      aims at facilitating the collaboration of students and researchers 
      on the development of their FEM codes.
%\vspace{-3mm}
\item {\em Broadening the participation of groups underrepresented in science, mathematics, 
      engineering and technology:}
      In 2004-2008, the PI played a leading role in establishing a new interdisciplinary 
      Ph.D. program in Computational Science at the University of Texas at El Paso (UTEP),
      one of the largest Hispanic-serving institutions in the nation.
%\vspace{-3mm}
\item {\em Service to the scientific and engineering community:} 
The PI organizes international conferences 
{\em European Seminar on Coupled Problems (ESCO 2008, ESCO 2010)}
and {\em Finite Element Methods in Engineering and Science 
(FEMTEC 2006, FEMTEC 2009)}, and he serves as referee for many scientific journals including 
SIAM J. Numer. Analysis, J. Comput. Phys., Math. Comput. Sim., 
J. Comput. Appl. Math., Comput. Methods Appl. Math. Engrg., Int. J. Numer. Appl. Math., 
Appl. Math. He also reviews regularly manuscript proposals for CRC Press and 
J. Wiley \& Sons. 
%\vspace{-8mm}
%Member of three technical committees of IMACS. Member of editorial board of 
%Adv. Appl. Math. Mech., Surv. Math. Appl., and Acta Technica. 
\end{itemize}

\subsubsection*{Collaborators and other affiliations}

{\em Collaborators during the 48 months preceding this proposal submission}:
I. Dolezel (Academy of Sciences of the Czech Republic, Prague),
G. Hansen (Idaho National Laboratory),
V. Kreinovich (UTEP), 
D. Kuzmin (University of Houston), 
A. Pownuk (UTEP),
J. Ragusa (Texas A\&M University), 
T. Vejchodsky (Academy of Sciences of the Czech Republic, Prague).
{\em Graduate and Postdoctoral Advisors, Thesis Advisors, 
Postgraduate Scholar Sponsors}:
M. Feistauer (Charles University, Prague, Czech Republic),
H.W. Engl (Johannes-Kepler University, Linz, Austria),
L. Demkowicz (ICES, University of Texas, Austin),
P. Kloucek (Rice University).

\subsubsection*{Students supervised}

O. Certik (Ph.D. expected in 2012), 
L. Dubcova (Ph.D. expected in 2010), 
P. Kus (Ph.D. expected in 2010), 
J. Cerveny (Ph.D. expected in 2010), 
M. Zitka (M.S. 2005, Ph.D. 2008), 
D. Andrs (Ph.D. 2008), 
J. Avila (M.S. 2006), and
S. Vyvialova (M.S. 2005). The total number of 
graduate students supervised by the PI is 8.
A postdoctoral fellow sponsored by the PI, David Andrs,  
became staff member of Idaho National Laboratory in January 2010. 
The PI has been mentoring several undergraduate students and 
has a strong record of working with underrepresented 
minority students at the University of Texas at El Paso (UTEP). 


%\subsubsection*{Awards and Honors}
%Bernardo Bolzano Prize (M.S. thesis, 1996),
%Josef Hlavka Prize (Ph.D. dissertation, 1999), 
%Babu\v ska Prize (Ph.D. dissertation, 1999), 
%TICAM Postdoctoral Fellowship Award (UT Austin, 2001 - 2002),
%W.M. Rice Postdoctoral Fellowship Award (Rice University, 2002 - 2004),
%Prize of the First Degree by the President of the 
%Czech Technical University (monograph {\em P. \v Sol\'{\i}n, K. Segeth, 
%I. Dole\v zel: Higher-Order Finite Element
%Methods (CRC Press/Chapman \& Hall, 2003}), 2004),
%The article {\em P. \v Sol\'{\i}n, K. Segeth: Non-Uniqueness of 
%Almost-Unidirectionel Inviscid Compressible Flow, 
%Appl. Math. 49, pp. 247 - 268, 2004} was selected as the best
%paper published in Appl. Math. in 2004.

\end{document}
\newpage

\subsection*{Role of Pavel \v Sol\'{\i}n in the proposed project}

P. \v Sol\'{\i}n's expertise is in the theory of partial differential 
equations (PDEs), classical finite element methods (FEM),
and modern hierarchic higher-order finite element methods ($hp$-FEM).
P. \v Sol\'{\i}n will collaborate simultaneously with V. Kreinovich 
and R. Muhanna and use their expertise to incorporate 
the best known interval techniques into the $hp$-FEM. The modular
multi-physics $hp$-FEM system HERMES will be used for implementation
and testing purposes. Moreover, P.  \v Sol\'{\i}n is highly interested 
in applying interval techniques to solve challenging open theoretical  
problems related to the $hp$-FEM, such as the discrete maximum
principles and others. Together with J. Chessa, P. \v Sol\'{\i}n
will investigate the application of the $hp$-FEM equipped with interval
technology to nonlinear problems lying in J. Chessa's area of 
expertise. 

On the practical side, P. \v Sol\'{\i}n is the PI of the project 
and he will have the responsibility for managing the project and 
take reporting actions. Locally on UTEP he will coordinate the 
communication between V. Kreinovich, J. Chessa and the students,
as well as the communication with R. Muhanna who is at Georgia Tech.
P. \v Sol\'{\i}n will also advise the postdoctoral associate
and one graduate and one undergraduate students who will be supported
from the proposed project. Three other graduate students 
will be advised by V. Kreinovich, J. Chessa, and R. Muhanna,
but P. \v Sol\'{\i}n 
will see that they work in a coordinated way with the other students
and the rest of the team.

\end{document}
