%&csplain

\input eplain   

%\voffset=-1.5cm
%\hoffset=-1.5cm
%\advance \vsize by 3cm
%\advance \hsize by 3cm

\magnification\magstep1
%\chyph
%\csaccents
%\raggedbottom

\newcount\eqcount
\def\incno#1{\global\advance\eqcount1\definexref{#1}{\the\eqcount}{eq}}
\def\no#1{\incno{#1}\eqno{(\the\eqcount)}}
\def\nno#1{\incno{#1}&{(\the\eqcount)}}
\def\rno#1{(\refn{#1})}


\def\mydots{\leaders\hbox{$\mkern1.5mu.\mkern1.5mu$}\hfill}
\def\tocsecentry#1#2#3{\line{\bf#2 #1\thinspace\mydots#3}}
\def\tocsubentry#1#2#3{\line{\quad#2 #1\thinspace\mydots#3}}

\newcount\secno
\newcount\subno
\def\secskip{\vskip 18pt plus 6pt minus 6pt}

\font\titlefont=\fontname\tenbf\space scaled\magstep2
\font\subfont=\fontname\tenbf\space scaled\magstep1
\font\subsubfont=\fontname\tenbf\space scaled\magstep0
\font\progfont=cstt8
\def\title#1{\centerline{\titlefont #1}\bigskip}
\def\sec#1\par{\advance\secno by1\subno=0\secskip\goodbreak\noindent%
{\subfont \the\secno. #1}%
\writenumberedtocentry{sec}{#1}{\the\secno.}\medskip\nobreak\noindent}
\def\sub#1\par{\advance\subno by 1\medskip\goodbreak\noindent%
{\edef\n{\the\secno.\the\subno}{\subsubfont\n\ #1}%
\writenumberedtocentry{sub}{#1}{\n}}\par\nobreak\noindent}

\newcount\figno
\def\dref#1{\global\advance\figno by 1\definexref{#1}{\the\figno}{}\the\figno}

\font\twelverm=csr12
\def\d{{\rm d}}

\def\psfig#1#2#3{\medskip\noindent\epsfxsize=\hsize\epsfbox{#1}
\centerline{Obr. \dref{#2}: #3}\medskip}

\def\y#1{(#1 let)}
\def\p#1#2{ {\partial#1\over\partial#2} }
\def\pp#1#2{ {\partial^2#1\over\partial#2^2} }
\def\degrees{^\circ}
%\def\Res#1#2{{\rm Res}({#1};{#2})}
%\def\Res#1#2{{\rm Res}_{#2}{#1}}
\def\Res#1#2{\mathop{\rm Res}_{#2}{#1}}

%end of macro definitions

\title{FEM}

\centerline{Ond�ej �ert�k}
\centerline{May, 2005}

\sec{Introduction}

There is a lot of theoretical information about FEM in the literature or the
internet, here I only show how to get started - from the start (the PDE
equation) to the end (matrix elements in the linear equations system for the
solution) and how to impose boundary conditions. I do not bother about any
assumptions or whatever, these can be found in the literature.

I am not saying this is the only way of solving PDE using FEM, I just wanted to
write down (for me and maybe others) what exactly I am doing in my program.

\sec{Poisson Equation}

$$-\nabla\cdot(\lambda\nabla u)=f\,.$$

\sub{Weak Formulation}

We multiply both sides with a test function $v$
$$-\left(\nabla\cdot(\lambda\nabla u)\right)v=fv\,,$$
and integrate over the whole volume
$$\int-\left(\nabla\cdot(\lambda\nabla u)\right)v\,\d V=\int fv\,\d V\,,\no{1}$$
and using the vector identity
$$-\left(\nabla\cdot(\lambda\nabla u)\right)v=\lambda\nabla u\cdot
\nabla v - \nabla\cdot\left(\lambda(\nabla u)v\right),$$
we rewrite the left hand side of \rno{1}
$$\int\lambda\nabla u\cdot\nabla v\,\d V=\int fv\,\d V+
\int\nabla\cdot\left(\lambda(\nabla u)v\right)\d V\,,$$
now we apply Gauss Theorem
$$\int\lambda\nabla u\cdot\nabla v\,\d V=\int fv\,\d V+
\oint\lambda(\nabla u)v\cdot{\bf n}\,\d S\,,$$
and rewriting $\nabla u\cdot{\bf n}\equiv{\d u\over\d n}$
$$\int\lambda\nabla u\cdot\nabla v\,\d V=\int fv\,\d V+
\oint\lambda{\d u\over\d n}v\,\d S\,,\no{w}$$
which is the weak formulation. The problem reads: find a function $u$ such that
\rno{w} holds for every $v$.

\sub{Linear Equations}

Now we substitute $\phi_i$ for $v$ and expand $u=\sum u_j\phi_j$
$$\int\lambda\nabla\phi_j\cdot\nabla\phi_i\,\d Vu_j=\int f\phi_i\,\d V+
\oint\lambda{\d u\over\d n}\phi_i\,\d S\,,\no{d}$$
which can be written in a matrix form
$$K_{ij}u_j=F_i\,,\no{m}$$
where
$$\eqalign{
K_{ij}&=\int\lambda\nabla\phi_j\cdot\nabla\phi_i\,\d V\,,\cr
F_i&=\int f\phi_i\,\d V+\oint\lambda{\d u\over\d n}\phi_i\,\d S\,.\cr
}$$
The matrix $K_{ij}$ is symmetric in this case, but generally it doesn't have to
be (ie. notice where the $i$'s and $j$'s are).

\sub{Finite Elements}

We decompose the domain into elements, so we need to compute the integrals over
one finite element:
$$\eqalign{
K_{ij}^{E}&=\int\lambda\nabla\phi_j\cdot\nabla\phi_i\,\d V^{E}\approx
\sum_{q=0}^{N_q-1}\lambda(x_q)\,\nabla\phi_i(x_q)\cdot\nabla\phi_j(x_q)\,
w_q|\det J(\hat x_q)|\,,\cr
f_i^{E}&=\int f\phi_i\,\d V^{E}\approx
\sum_{q=0}^{N_q-1}f(x_q)\,\phi_i(x_q)\, w_q|\det J(\hat x_q)|\,.\cr
}$$
The surface integrals are computed accordingly.

\sub{Boundary Conditions}

The Neumann boundary condition is already taken into
account in \rno{m} - it's the surface integral containing
${\d u\over\d n}$, so we compute this integral over every element, where 
${\d u\over\d n}\neq0$.

The Dirichlet boundary condition is imposed by adding the following equation to
\rno{d} for every element we want to prescribe the value of $u$:
$$P\int\phi_i\phi_j\,\d Su_j=P\int u_0\,\phi_i\,\d S\,,$$
where $P$ is some big number (penalty) and $u_0$ is the value of $u$ on the
boundary. Then the matrices are (on this element) as follows:
$$\eqalign{
K_{ij}&=\int\lambda\nabla\phi_j\cdot\nabla\phi_i\,\d V+
	P\int\phi_i\phi_j\,\d S\,,\cr
F_i&=\int f\phi_i\,\d V+\oint\lambda{\d u\over\d n}\phi_i\,\d S+
	P\int u_0\,\phi_i\,\d S\,.\cr
}$$

The second possibility to impose the Dirichlet boundary condition is to set the
$i$ row in RHS to $u_0 P$ and the element $i,i$ in the stiffness matrix to
$P$.

The first way is more general and more precise because we don't bother about
any nodes, we just need the value of $u_0(x,y,z)$ on the whole boundary (and
take into account only the $u_0$'s in the Gauss points). The second way is
better if we just want to set the value of $u$ in some particular nodes.

\sec{Heat Equation}

$$c\p{u}{t}-\nabla\cdot(\lambda\nabla u)=f\,.$$

\sub{Weak Formulation}

In exactly the same way as for the Poisson equation we derive the weak form:
$$\int c\p{u}{t}v\,\d V+
\int\lambda\nabla u\cdot\nabla v\,\d V=\int fv\,\d V+
\oint\lambda{\d u\over\d n}v\,\d S\,,$$

\sub{Linear Equations}

As usual after the substitution $\phi_i$ for $v$ and 
expansion $u=\sum u_j\phi_j$
$$\int c\phi_j\phi_i\,\d V\p{u_j}{t}+
\int\lambda\nabla\phi_j\cdot\nabla\phi_i\,\d Vu_j=\int f\phi_i\,\d V+
\oint\lambda{\d u\over\d n}\phi_i\,\d S\,,$$
and a matrix form
$$C_{ij}\p{u_j}{t}+K_{ij}u_j=F_i\,,$$
where
$$\eqalign{
C_{ij}&=\int c\phi_j\phi_i\,\d V\,,\cr
K_{ij}&=\int\lambda\nabla\phi_j\cdot\nabla\phi_i\,\d V\,,\cr
F_i&=\int f\phi_i\,\d V+\oint\lambda{\d u\over\d n}\phi_i\,\d S\,.\cr
}$$

\sec{Wave Equation}

$${1\over c^2}\pp{u}{t}-\nabla^2u=0\,.$$

\sub{Weak Formulation}

In exactly the same way as for the Poisson equation we derive the weak form:
$$\int{1\over c^2}\pp{u}{t}v\,\d V+
\int\nabla u\cdot\nabla v\,\d V=
\oint{\d u\over\d n}v\,\d S\,,$$

\sub{Linear Equations}

As usual after the substitution $\phi_i$ for $v$ and 
expansion $u=\sum u_j\phi_j$
$$\int{1\over c^2}\phi_j\phi_i\,\d V\pp{u_j}{t}+
\int\nabla\phi_j\cdot\nabla\phi_i\,\d Vu_j=
\oint{\d u\over\d n}\phi_i\,\d S\,,$$
and a matrix form
$$M_{ij}\pp{u_j}{t}+K_{ij}u_j=F_i\,,$$
where
$$\eqalign{
M_{ij}&=\int{1\over c^2}\phi_j\phi_i\,\d V\,,\cr
K_{ij}&=\int\nabla\phi_j\cdot\nabla\phi_i\,\d V\,,\cr
F_i&=\oint{\d u\over\d n}\phi_i\,\d S\,.\cr
}$$

\sec{Libmesh}

Libmesh is a promising finite element method code. Web: {\tt libmesh.sf.net}.
It's clearly written and has almost everything or can easily be extended.

\sub{Equation systems}

We create one instance of {\tt EquationSystems} class and every equation we
want to solve (Poisson, Heat, \dots) must be added to this instance by {\tt
addsystem()} method.

One system (for example Poisson equation) is handled by a class {\tt System} or
it's children. The {\tt System} class has a method {\tt assemble()}, which just
calls the user assembly method - it's up to the user what he does.  

{\tt ExplicitSystem} contains {\tt rhs} vector and method {\tt solve()}, which
just calls {\tt System::assemble()}. 

{\tt ImplicitSystem} contains {\tt matrix} matrix and a method {\tt assemble()}
which clears {\tt matrix} and {\tt rhs} and calls {\tt 
System::assemble()}.

{\tt LinearImplicitSystem} contains a method {\tt solve()}, which calls {\tt
ImplicitSystem::assemble()} and then solves the equation $Ax=b$, where $A$ is
{\tt matrix} and $b$ is {\tt rhs}. $A$ (= {\tt matrix}) is often referred to as
a "system matrix".

{\tt NewmarkSystem} implements Newmark integration scheme which solves:
$$M_{ij}\pp{u_j}{t}+C_{ij}\p{u_j}{t}+K_{ij}u_j=F_i\,,$$
where $M_{ij}$ is called {\it mass} matrix, $C_{ij}$ {\it damping} matrix,
$K_{ij}$ {\it stiffness} matrix and $F_i$ {\it force} vector. 
See example 8. (The Newmark System and the Wave Equation) for usage.
Reimplements {\tt assemble()}, which calls {\tt
LinearImplicitSystem::assemble()}, then {\tt compute\_matrix()} and {\tt
initial\_conditions()}. Subsequent calls to this method do nothing. {\tt
compute\_matrix()} does: 
$$A=a_0M+a_1C+K\,.$$ 
{\tt initial\_conditions()} clears {\tt displacement}, {\tt velocity} and {\tt
acceleration} vectors and calls user initialization method.
Uses {\tt LinearImplicitSystem::solve()}, which calls our new {\tt assemble()}
function. The user assembly method is supposed to change $M$, $C$, $K$ and $F$.
The user should repeatedly do: {\tt update\_rhs()}, update 
Dirichlet boundary conditions, {\tt solve()}, save solution, 
{\tt update\_u\_v\_a()}.
{\tt update\_rhs()} does:
$$b=F+M*(a_0u+a_2u'+a_3u'')+C*(a_1u+a_4u'+a_5u'')\,.$$ 
{\tt update\_u\_v\_a()} does:
$$u''(i+1)=-a_3u''(i)+a_0u(i+1)-a_0u(i)-a_2u'(i)\,,$$ 
$$u'(i+1)=u'(i)+a_6u''(i)+a_7u''(i+1)\,.$$ 
$$\eqalign{
a_0&={1\over \alpha (\Delta t)^2}\cr
a_1&={\delta \over \alpha \Delta t}\cr
a_2&={1\over \alpha \Delta t}\cr
a_3&={1\over 2\alpha} -1\cr
a_4&={\delta \over \alpha}-1\cr
a_5&={\Delta t\over 2}{\delta\over \alpha -2}\cr
a_6&={\Delta t(1-\delta)}\cr
a_7&={\delta \Delta t}\cr
}$$

The Newmark time integration scheme is 
$$\eqalignno{
u_{n+1}&=u_n+\Delta t u_n' + {1\over2}(\Delta t)^2\left(
	\left(1-2\alpha\right)u_n''+2\alpha u_{n+1}'' 
	\right)\nno{un1}\cr
u_{n+1}'&=u_n'+\Delta t\left((1-\delta)u_n''+\delta u_{n+1}''\right)\nno{un2}\cr
}$$
As you can see, the {\tt update\_u\_v\_a()} method computes $u_{n+1}''$ from
\rno{un1} and $u_{n+1}'$ from \rno{un2}.
This scheme in unstable for $\delta<{1\over2}$. For $\delta>{1\over2}$
it is conditional stable for $\alpha<{(\delta+{1\over2})^2\over4}$,
uncoditional otherwise. 

Standard {\tt libmesh} values are $\alpha={1\over4}$ and $\delta={1\over2}$.
This choice of parameters corresponds to a trapezoidal rule (unconditionally
stable scheme in linear analyses). But for a wave equation from my own
experience the default values are unstable (even smaller timestep doesn't help)
only setting $\alpha=1.0$ and $\delta=2.0$ helped. Update: ok I already now
what's wrong: there is some bug in laspack library, trying petsc works
perfectly with the default parameters.

how it works: from \rno{un1} we express $n_{n+1}''$
$$u_{n+1}''={1\over\alpha(\Delta t)^2}u_{n+1}-\left({1\over2\alpha}-1\right)u_n''
-{1\over\alpha(\Delta t)^2}u_n-{1\over\alpha\Delta t}u_n'$$
We have this equation
$$Mu_{n+1}''+Ku_{n+1}=F$$
and substitute for $u_{n+1}''$ and we get
$$\left({M\over\alpha(\Delta t)^2}+K\right)u_{n+1}=F+
M\left(\left({1\over2\alpha}-1\right)u_n''
+{1\over\alpha(\Delta t)^2}u_n+{1\over\alpha\Delta t}u_n'\right)$$

\bye
